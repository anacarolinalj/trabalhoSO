\documentclass[a4paper]{article}

%% Language and font encodings
\usepackage[english]{babel}
\usepackage[utf8x]{inputenc}
\usepackage[T1]{fontenc}

%% Sets page size and margins
\usepackage[a4paper,top=3cm,bottom=2cm,left=3cm,right=3cm,marginparwidth=1.75cm]{geometry}

\title{Implementação de um Pseudo-SO}
\author{Ana Carolina Lopes, Jorge Luiz Andrade}
\date{}

\begin{document}
\maketitle

\section{Introdução}

\section{Ferramentas e Linguagem utilizada}

\section{Solução desenvolvida}
\subsection{Módulo de Processos}

\subsection{Módulo de Filas}

\subsection{Módulo de Memória}
	O módulo de memória conta basicamente com um \emph{array} de inteiros com 1024 posições por padrão. Cada posição desse \emph{array} corresponde a um bloco da memória, com seu conteúdo sendo igual ao \textbf{pid} do processo alocado, ou a \textbf{-1} caso o bloco esteja livre.
	
	Além disso, o módulo considera que a região para processos em tempo real possui 64 blocos, sendo os 960 blocos restantes reservados para processos de usuário.
	
	O módulo possui apenas dois métodos, \textbf{alocaBloco} e \textbf{liberaBloco}. \textbf{AlocaBloco} aloca uma porção contígua de memória para um processo, respeitando a região reservada para cada tipo de processo(tempo real ou usuário), retornando o índice do início do bloco de alocação. Caso não encontre blocos contíguos suficientes para o processo a alocação não é feita e o método retorna -1. \textbf{LiberaBloco} libera toda a região de memória anteriormente alocada para um processo.

\subsection{Módulo de Recursos}

\section{Dificuldades encontradas}

\section{Divisão do trabalho}
\bibliographystyle{alpha}
\bibliography{sample}

\end{document}